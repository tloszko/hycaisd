\chapter{Lista 3}

\section{} %1
Zadanie to polega na skonstruowaniu szybkiego algorytmu obliczania największego wspólnego dzielnika dwóch dodatnich liczb całkowitych $a$ i $b$. Przed podaniem algorytmu musimy udowodnić podane właściwości:

\begin{equation*}
gcd(a,b) =
 \begin{cases}
 2 \cdot gcd( \frac{a}{2}, \frac{b}{2}) & \text{a,b są parzyste;} \\
 gcd(a,\frac{b}{2}) & \text{a jest nieparzyste, b jest parzyste;} \\
 gcd(\frac{a-b}{2},b) & \text{a,b są nieparzyste.}
 \end{cases}
 \end{equation*}

\subsection{a)}
Jeżeli $a$ i $b$ są parzyste, $2$ na pewno jest ich wspólnym dizelnikiem. Jeżeli $a$ jest nieparzyste i $b$ jest parzyste, wiemy że $b$ dzieli się przez 2, a $a$ nie. Więc $gcd(a,b)$ pozostaje takie same dla $a$ i $b/2$. Ostatnia własność wynika z faktu, że dla nieparzystych $a$ i $b$, $(a-b)$ będzie parzyste.Ponieważ $gcd(a-b,b) = gcd(a,b)$ oraz $(a-b)$ jest teraz parzyste, możemy zastosować drugą własność.

\subsection{b) algorytm rekurencyjny}
\begin{lstlisting}
procedure gcd(a, b)
Input: Two n-bit integers a,b
Output: GCD of a and b
if a = b:
	return a
else if (a is even and b is even):
	return 2*gcd(a/2, b/2)
else if (a is odd and b is even):
	return gcd(a, b/2)
else if (a is odd and b is odd and a > b):
	return gcd((a-b)/2, b)
else if (a is odd and b is odd and a < b):
	return gcd(a,(b-a)/2)
\end{lstlisting}
\subsection{c) złożoność}
Założmy, że $a$ i $b$ są n-bitowymi liczbami. Rozmiar $a$ i $b$ wynosi $2n$ bitów. Wszystkie z czterech ifów, oprócz przypadku gdzie $a$ jest nieparzyste i $b$ jest parzyste, zmniejsza rozmiar $a$ i $b$ do $2n - 2$ bitów, gdzie wcześniej wymieniony przypadek zmniejsza ilość bitów do $2n-1$. Każda z operacji wykonuje się w czasie stałym ponieważ dzielimy lub mnożymy przez 2. Dla dwóch przypadków z odejmowaniem, mamy odejmowanie dwóch n-bitowych liczb (złożoność wynosi $c\cdot n$ gdzie n jest wielkością operandu). Zatem najgorszy przypadek czwartego ifa algorytmu przedstawimy jako:

\begin{equation*}
\begin{split}
T(2n) & = T(n-1) + cn \\
T(2n -1) & = T(2n - 2) + cn \\
T(2n - 2) & = T(2n - 3) + c(n-1)  \text{   oba operandy mają długość $n-1$} \\
T(2n - 3) & = T(2n - 4) + c(n-1) \\
... &  \\
T(2) &=  T(1) + c
\end{split}
\end{equation*}

Podstawieniami możemy zapisać $T(2n)$ jako:
$$ T(2n) = 2c\cdot \sum_{i=1}^n i $$
co daje nam $O(n^2)$ co w porównaniu do $O(n^3)$ czasu działania algorytmu euklidesa jest szybsze.
\section{} %2
\href{https://github.com/edhell/hycaisd/raw/master/listy/2012/solutions/solutions/lista_3_zadanie_2.pdf}{PDF}

\section{} %3
Średnicą drzewa nazywamy największą odległość między parami wierzchołków. Będziemy rozważali drzewa ukorzenione w wierzchołku $r$. Wprowadźmy pewne oznaczenia: analizujemy wierzchołek $v$, który ma $m$ synów ponumerowanych od $0$ do $m-1$. Niech $T(v)$ oznacza poddrzewo zaczepione w wierzchołku $v$, natomiast $s[v]$ jego rozmiar (liczbę wierzchołków).

Dla każdego wierzchołka będziemy pamiętali dwie liczby: $tak[v]$ - długość najdłuższej ścieżki z $T(v)$, która kończy się w wierzchołku $v$ oraz $nie[v]$ - długość najdłuższej ścieżki z $T(v)$, która nie kończy się w $v$. Średnica drzewa jest równa $max(tak[r], nie[r])$.

Tablice $tak$ i $nie$ wyliczamy za pomocą następującej rekurencji:

$$ tak[v] = max_{0\leq i < m}( 1 + tak[i] ) $$
$$ nie[v] = max\big( max_{0\leq i < m}( nie[i] ), max_{0\leq i < m}(tak[i]), max_{0\leq i <j<m}( tak[i]+2+tak[j]) \big) $$

Dla wierzchołka $v$ obliczenia zabierają czas $O(m)$ (wyszukanie dwóch największych elementów w tablicy da się zrobić liniowo). Zatem dla całego drzewa czas wynosi $O(n)$.

\section{} %4

Ukorzeniamy drzewo w pewnym wierzchołku. Nazwijmy ten wierzchołek $u$, a jego dzieci $v_1, v_2, \ldots , v_n$

Zauważamy, że ścieżka długości $C$ albo przechodzi przez ten korzeń albo nie przechodzi:

a) jeśli ścieżka nie przechodzi przez korzeń, to znajduje się z którymś z $n$ niższych poddrzew z wierzchołkami $v_1, v_2, \ldots, v_n$ (innymi słowy dziel i zwyciężaj).

b) jeśli ścieżka przechodzi przez korzeń to albo wychodzi bezpośrednio z korzenia i przechodzi przez jakiś wierzchołek $v_i$ i idzie niżej (czyli ścieżka jedno-fałowa) albo łączy ze sobą dwa wierzchołki $v_i$ i $v_j$ i z dwóch stron idzie niżej (nazwijmy taki typ ścieżki dwu-fałowy). Dodatkowo musimy zapewniać, że połączone ścieżki są długości $C$.


Zaczniemy od końca, czyli od punktu drugiego. Mamy już jakiś korzeń i chcemy policzyć, ile ścieżek przechodzący przez ten korzeń jest długości $C$. Zauważmy, że łatwo obliczyć odległość dowolnego wierzchołka do korzenia (wystarczy DFS). Teraz stwórzmy sobie strukturę, która potrafi nam coś zrobić:

$ile(d, 1..i) $ liczba ścieżek dochodzący do korzenia, które:
a) są długości $d$
b) a także przechodzą przez któryś z wierzchołków $v_1, \ldots, v_i$

$wstaw(d, i+1) $ dodaje informację do bazy o ścieżek do korzenia długości $d$, która, przechodzi przez wierzchołek $v_{i+1}$.

Drzewo łatwo sobie na rysować na kartce. Jeśli popatrzymy na ścieżkę, to jakiś fragment idzie tak jakby z lewej na prawo (lub na odwrót, zależy jak popatrzymy). Nas będzie interesować ten wierzchołek na końcu ścieżki, który jest najbardziej na prawo, oznaczmy go przez $x$. Niech $d(x)$ to odległość do korzenia. Po drodze zauważamy, że ścieżka przechodzi przez wierzchołek $v_j$, a potem idzie jeszcze bardziej w lewo przez jakiś wierzchołek $v_i$, czyli musi być, że $i<j$. Łatwo zauważyć, że wszystkich ścieżek długości $C$, a kończących się na $x$ jest dokładnie
tyle ile jest "lewych" ścieżek dochodzących do korzenia, ale długości $C-d(x)$, czyli
$ile(C - d(x), 1..j-1)$.

Można się domyślić, że algorytm będzie wyglądał tak:

1. początkowo $ile(d, 1..0) = 0$, bo wiemy, że nic nie wiemy, za wyjątkiem $ile(0, 1..0) = 1$, bo jest jedna ścieżka długości $0$.

2. Dla kolejnych $i = 1, 2, \ldots, n$, wykonujemy poniższe czynności:

a) obliczamy odległości do korzenia wierzchołków w poddrzewie $v_i$

b) do wyniku dodajemy $ile(C - d(x), 1..(i-1))$ ($x$ - wierzchołek w poddrzewie)

c) uaktualnimy $ile$ poprzez wywołanie $wstaw(d(x), i+1)$ dla wszystkich $x$ w poddrzewie $v_i$

Łatwo zauważyć, że parametr $i$ jest niepotrzebny. Całą strukturę łatwo można stworzyć na jakimś drzewie zrównoważonym (czyli mapie). Wykonujemy dokładnie $O(m)$ zapytań, które kosztują nas $O(log m)$, bo w mapie może być co najwyżej $m$ różnych liczb, bo jest co najwyżej $m$ ścieżek do korzenia, zatem złożoność wynosi $O(m log m)$.

Teraz powróćmy do punktu pierwszego, cały algorytm będzie wyglądał tak:

1. Znajdź jakiś fajny wierzchołek, który dzieli graf na "połowę". (zrobimy to w czasie $O(m)$)

2. Policz ile jest ścieżek przechodzących przez korzeń o długości $C$ (opisane wyżej). (Czas $O(m log m)$)

3. Wykonaj rekurencyjnie to samo, ale na poddrzewach.


Teraz jak znaleźć fajny wierzchołek. Okazuje się, że istnieje taki wierzchołek, że jeśli go ukorzenimy, to każde poddrzewo jest wielkości co najwyżej $\lceil n/2 \rceil$. A jak go znaleźć: bierzemy wierzchołek $v$, jeśli nie spełnia założeń, to wchodzimy do największego poddrzewa (dowód poprawności jest podobny, nie wprost, pokazujemy, że można zejść niżej).

Całkowita złożoność wynosi $O(m \log^2 m)$.

\section{} %5
\subsection{a)}
W wektorze pamiętamy pierwszą kolumnę oraz pierwszy wiersz bez pierwszego wyrazu. Wektor ten ma rozmiar $2n - 1$, więc dodawanie dwóch takich wektorów mamy w $O(n)$.
\subsection{b)}
Macierz Toeplitza ma następującą postać blokową:
$$ T = \left[ \begin{matrix}
					A & B \\
					C & A
				\end{matrix} \right]$$

Zadanie polega na pomnożeniu macierzy T przez wektor blokowy $ T = \left[ \begin{matrix}x \\ y\end{matrix} \right] $ w czasie mniejszym niż $n^2$. Korzystając z dwóch (wzajemnie dualnych) obserwacji:

$$Ax = A(x + y - y) = A(x+y) - Ay$$

$$Ay = A(x + y - x) = A(x+y) - Ax$$


mamy:


\begin{equation*}
\begin{split}
T = \left[ \begin{matrix}A & B\\ C & A\end{matrix} \right] \cdot \left[ \begin{matrix} x \\ y\end{matrix} \right] = \left[ \begin{matrix} Ax + By\\ Cx + Ay\end{matrix} \right] = \\
= \left[ \begin{matrix}A(x+y)-Ay + By\\ Cx + Ay\end{matrix} \right] = \left[ \begin{matrix}A(x+y)+(B - A)y\\ Cx + Ay\end{matrix} \right] = \\
= \left[ \begin{matrix}A(x+y)+(B - A)y\\ Cx + A(x+y) - Ax \end{matrix} \right] =  \left[ \begin{matrix}A(x+y)+(B - A)y\\ A(x+y) + (C-A)x\end{matrix} \right]
\end{split}
\end{equation*}

I zauważamy, że złożoność czasowa to $T(n) = 3 T(n/2) + O(n)$ gdzie $O(n)$ zamyka sumę liniowych czasów wszystkich dodawań.

Rozwiązując typowe równanie rekurencyjne mamy $T(n) = O(n^{log_2 3}) < O(n^2)$.


\section{} %6
\section{} %7
